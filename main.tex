\documentclass[notitlepage,12pt]{article}

% Packages to use
\usepackage{setspace}
\usepackage{url}
\usepackage[table]{xcolor}
\usepackage{booktabs}
\usepackage{float}
\usepackage{lipsum}
\usepackage{graphicx}
\usepackage{natbib}
\doublespacing
\usepackage[margin=1in]{geometry}
\usepackage[english]{isodate}
\usepackage{fancyhdr}
\pagestyle{fancy}
\rhead{S. Hogan-H, J. Lee}
\lhead{Contemporary Monetary Policy, the Taylor Rule and the Recession}

% Author
\author{Senan Hogan-H, Jiwon Lee\footnote{This paper was completed in accordance with requirements for the Claremont Graduate University class ``Time-Series Econometrics," Spring 2017.  We are grateful for the comments and guidance of Pierangelo De Pace, Pomona College Department of Economics.} \\ Pomona College}

% Title
\title{Contemporary Monetary Policy, the Taylor Rule and the Recession: \\
\Large{Testing a Reaction Function for the Federal Reserve 1987--2014 }} 
\date{14 May 2017}


%%% BEGIN %%%
\begin{document}

\clearpage\maketitle
\thispagestyle{empty}
% Abstract
\begin{abstract}
This paper tests a model for the Federal Reserve's Reaction Function, the monetary policy response to economic developments in the context of a Taylor Rule and a Taylor with discretion for a time series context.  We consider how the reaction function has changed over two distinct time periods: the tenures of Chairmen Greenspan and Bernanke, updating the methods and results of \cite{judd1998taylor} to a new era of monetary policy.  We find that a Taylor-rule framework is an effective way to analyse and consider monetary policy for Chairman Greenspan's tenure; we find it is not effective for analysis during Chairman Bernanke's tenure.  We present reasoning for what the Great Recession and the monetary policy environment means for a Taylor rule and the Federal Reserve's approach to monetary policy.
\end{abstract}

% INTRODUCTION
\newpage
\setcounter{page}{1}
\section{Introduction}
Over the past several years, monetary policy reaction functions have considerably attracted the attention of macroeconomists. Within a macroeconomic model, the reaction function can be used to evaluate the actions and policy of central banks, and through these estimations, researchers have aimed to gain insight into the Federal Reserve's behavior. The most popular tool that has been used in estimating the Fed's reaction function is the Taylor Rule. Created by John Taylor (1993), the Taylor Rule is a simple linear rule that captures the dependence of the interest rate from the changes of the output and the inflation. While, the Taylor rule has been highly utilized in various empirical studies of monetary policy and theoretical framework for understanding the Federal Bank's reaction function, its specification has been recently questioned by several macroeconomists. Thus, there have been several efforts to modify the Taylor Rule to better estimate the Reserve's reaction functions.

This paper tests a model for the Federal Reserve's Reaction Function -- the monetary policy response to economic developments -- from 1987 to 2014 in the context of a Taylor Rule and a Taylor with discretion for a time series context. In our analysis, we utilize a variant of Taylor's simple specification in understanding the Federal Reserve's reaction functions. Our analysis extends that of Judd \& Rudebusch (1998), who suggest several alternative specifications to the Taylor Rule to better capture the inconsistency in the Federal Reserve’s reactions. First, rather than taking parameters as given, we econometrically estimate both the simple and the dynamic Taylor rule and the corresponding reaction function weights. Second, we account for the co-integrating relationship that might stem from the Federal Reserve’s gradual responses over time and utilize an estimation method that takes the non-stationarity into account, namely an error-correction model. Lastly, we consider the changes in the Federal Reserve Chairmanship as an important and identifiable way to understand the changes in the composition of the Federal Open Market Committee, which might shift the preferences and conceptions of the policymakers involved. Thus, we separate our analysis by the different chairmen regimes, more specifically, those of Chairman Greenspan and Bernanke. We find that a Taylor-rule framework is an effective way to analyze and the Reaction Function of the Reserve before Chairman Bernanke's tenure, yet not effective afterwards during the Great Recession and its aftermath.

	Section 2 provides a literature review on the Reserve's Reaction function and the Taylor Rule. Section 3 describes the econometric specifications and methods used in this paper, along with the description of data. Section 4 includes the results, and the final section concludes.


% LIT REVIEW
\section{Literature Review}
Macroeconomists have long been interested in modeling the Federal Reserve’s “reaction function” - that is, a model of how the Reserve alters monetary policy in response to economic developments (Judd \& Rudebusch, 1998). The Fed's reaction function plays an important role in a wide variety of macroeconomic analyses. The Reserve's reaction function can provide a basis for forecasting changes in the Fed's policy instrument, namely short term interest rates, and it is an important element in evaluating the Fed's policy and the effects of their policy actions or economic shocks (Rabin, 2001). For these benefits of understanding the reaction function, several researchers have been committed to gain insight into central bank behavior through the estimations of these reaction functions. 

The most popular tool that has been used by macroeconomists in estimating the reaction function is the Taylor Rule. John Taylor (1993) made an important contribution to this field with a simple characterization of the Federal Reserve's monetary policy, and he provided a linear function of two key variables, inflation and output, as a simple rule of thumb for the Reserve's monetary policy. Since its introduction in his work ``Discretionary vs Policy Rule in Practice," the Taylor Rule has been highly utilized in various empirical studies of monetary policy and theoretical framework for understanding the Federal Bank's reaction function (Judd \& Rudebusch, 1998).

The Taylor Rule specifies that the real federal funds rate reacts to two variables: deviations of inflation from the target inflation and gap between real output and the potential output (Taylor, 1993). This simple rule relates a plausible Federal Reserve instrument, namely the short-run interest rate, to the goals of stabilizing inflation and output growth. The Taylor Rule's systematic and relatively straightforward approach has numerous advantages, such as serving as a useful guide to the Federal Open Market Committee when it sets monetary policy, helping set reliable parameters for people involved in the financial industry, and allowing the Reserve pro-actively send signals to the public (Kohn, 2007; Vitruk, 2014). \nocite{kohn2007john} \nocite{vitruk2014development} 

Several recent studies suggest that simple Taylor-type reaction functions were found to perform almost as well as optimal, forecast-based reaction functions that incorporate all the information available in the models examined. For example, Orphanides (2002) finds that the Fed's policies for many periods can be broadly interpreted in terms of the Taylor-rule framework with surprising consistency. Judd and Rudebusch (1998) also conclude that a Taylor-rule framework is a useful way to summarize key elements of monetary policy from 1970 to 1997. 

While the Taylor rule has reached widespread popularity, not all researchers agree that its framework provides a definite representation of the Fed's behavior. For example, \cite{khoury1990federal} surveys several empirical Reserve reaction functions from various studies and finds little consistency in the significance of various regressors in those reaction functions. Moreover, using a VAR model from 1984 to 2003, \cite{galbraith2007fed} conclude that the Reserve does not target inflation or react to ``inflation signals,” and rather reacts to the very ``real" signal sent by unemployment, suggesting that a ``baseless fear of full employment" is a principal force behind monetary policy. 

Accordingly, the Taylor Rule’s specification and its validity in estimating the Fed's reaction function have recently been a concern for some macroeconomists. Most importantly, several studies point out that the prior literature ignores the time series components of the variables included in the model. For instance, \cite{siklos2006estimating} show that if variables involved in the Taylor rule specification are integrated of order one -- that is have roots close to unity -- misspecified regressions would provide spurious results. \nocite{siklos2006estimating} Siklos's (2006) study demonstrates that the econometric properties of the model should be carefully investigated in order to use an estimated Taylor rule to evaluate the Fed's reactions or making statements about its preferences over time.

Despite the critiques on misspecification, most researchers agree that the original Taylor rule provides a useful benchmark for discussing the Fed's policy regimes. Thus, there has been an effort to modify the Taylor Rule to estimate the Fed's reactions better with alternative specifications. Consequently, an extensive number of variants of the basic Taylor Rule have been employed in several studies. For instance, some researchers (i.e. McCallum \& Nelson, 1999) \nocite{mccallum1999nominal} argue for the inclusion of lagged variables due to informational delays in central bank reaction; some (i.e. Orphanides, 2002) \nocite{orphanides2002unreliability} suggest to include forecasts of the regressors in order to capture the forward looking incentives of the central bank.  \cite{woodford2001taylor} suggests that a desirable rule is likely to require that the intercept be adjusted in response to fluctuations in the natural rate of interest, and should vary in response to a variety of real disturbances. He also suggests that an optimal rule should involve a commitment to ``history-dependent behavior," and indicates that more gradual adjustments of the interest rates should be considered.

Notably, Judd \& Rudebusch (1998) make a several modifications to the original Taylor Rule to provide a better estimation of the Fed’s policy over time. First, they econometrically estimates the reaction function weights rather than simply choosing parameters as in the original Taylor Rule. Furthermore, they utilize a variant of the basic Taylor Rule called the Dynamic Taylor Rule, a common alteration of Taylor’s Rule takes into account that the Reserve gradually and smoothly adjusts the interest rate over time. With the Dynamic Taylor Rule framework, they utilize an error-correction model to account for the presence of unit-roots and co-integrating relationships in the original model, and consequently to avoid spurious regressions. 

Furthermore, Judd \& Rudebusch (1998) suggest an effective way to estimate the Federal Reserve's reaction function over time, and they attempt to capture the inconsistent reactions of the Reserve by capturing the changes in the composition of the Federal Open Market Committee (FOMC). The main hypothesis is that understanding the changes in the composition of the FOMC would provide an explanation for the inconsistency of the Fed’s reaction function over time. They insist that such compositional changes may bring to the policymakers with different preferences and different conceptions of the appropriate operation and likely transmission of monetary policy (Judd \& Rudebusch, 1998). There are many things that influence this composition, but specifically, one of the more important and identifiable ones is the changes in Fed Chairmanship. Changes associated with different Chairmen may be exogenous, but there also may be an endogenous element that represents an adaption to ``lessons" learned from prior experiences (Judd \& Rudebusch, 1998). Accordingly, the estimates in their paper indicate that a Taylor-type reaction function seems to capture some important elements of monetary policy during Alan Greenspan's tenure to date as Federal Reserve Chairman. Further, they also provide evidence that support their hypothesis about the influence of the Chairman on the Fed's reaction function by showing that the dynamic Taylor-type reaction functions estimated during the Burns, Volcker, and Greenspan periods appear to have differed in important ways from one another (Judd \& Rudebusch, 1998). 

With the use of the Dynamic Taylor Rule and focusing on individual Reserve Chairmanship, Judd \& Rudebusch's (1998) investigation represent a step in the direction of uncovering the key elements and changes in Fed's behavior over time. By utilizing their analytical framework, we extend the analysis to the most recent chairmanship of Chairman Ben Bernanke (2006 - 2014) and compare his regime to that of Chairman Alan Greenspan (1984 - 2006). We are interested in analyzing the Bernanke period to investigate not only the possible changes in the Fed's reaction function from a new chairmanship, but also the Fed’s response to the Great Recession and its aftermath. Chairman Bernanke publicly argued that any mechanical formula, even if useful as a general policy guideline, could prove ineffective during especially tumultuous or unusual economic times (Miller, 2014). Thus, it is important to understand how the Fed, especially under Chairman Bernanke, reacts to deteriorating economic circumstances.

Furthermore, the comparison between the Chairmen Greenspan's and Bernanke's regime is remarkable because both chairmen are proponents of the Fed’s current adjustable policy (Vitruk, 2014). Chairman Greenspan has stated that the Taylor Rule is too uncertain to work on a consistent basis, and similarly, while Chairman Bernanke has emphasized that he does not oppose the Taylor Rule as a guidepost for prudent monetary policy, he opposes ``applying [it] in a mechanical way" (Vitruk, 2014). While allowing the Federal Reserve to employ a flexible monetary policy approach is prudent, regulators and economists have nevertheless accepted the Taylor Rule's use as an economic guideline rather than a strict mandate (Yellen 1996).\nocite{yellen1996monetary} Therefore, understanding that both chairmen are proponents of flexible monetary policy while accepting the Taylor Rule as a guidepost, we are interested in how the Fed's reactions in these time periods can be effectively analyzed by the Taylor-rule framework.


% METHODS
\section{Methods}
We use data for the US national economy, made available by the Federal Reserve Bank of St Louis, FRED Economic Data.  We consider the interest rate as the Federal Funds rate,\footnote{\cite{FEDFUNDS}} inflation given by the Consumer Price Index,\footnote{\cite{CPIU}} real Gross Domestic Product in Billions of Chained 2009 Dollars (seasonally adjusted annual rate),\footnote{\cite{GDPC1}} and real Potential Gross Domestic Product in Billions of Chained 2009 Dollars (seasonally adjusted annual rate).\footnote{\cite{GDPPOT}}  We use the CPI as our measure of inflation as it is the only inflation rate that the Federal Reserve has a publicly known rate goal for.  Table 1 presents summary statistics for our relevant variables.\footnote{The change in inflation (in percentage points) is calculated as the difference in four-quarter inflation from the first quarter to the last quarter
of the sample. End-of-sample inflation is average inflation over the final four quarters of the sample. Inflation is measured as the four-quarter change in CPI, and the interest rate is the federal funds rate.}

\begin{table}[H]
\caption{Summary Statistics: Interest Rates and Inflation}
\begin{tabular}{cccc}
                   & Long Sample & Greenspan & Bernanke     \\
                   & (1987.Q3---2014.Q1) & (1987.Q3--2006.Q1) & (2006.Q2--2014.Q1)    \\\midrule
Average real interest rate (\%)   & -0.79 & 0.15 & -3.10\\
Change in inflation  (\%)   &  -0.031  & -0.0092  & -0.087\\
Average inflation (\%) & 0.72 &0.98 &0.075\\
End-of-sample inflation (\%)  & -1.11 &1.49 &-1.11\\ \midrule \bottomrule
\end{tabular}
\end{table}

Equation 1 states the Taylor rule as outlined by \cite{taylor1993discretion}, where $a_{\pi }=a_{y} =\frac{1}{2}$, $r_{t}^{*}=2$.  $i_{t}$ is the nominal interest rate, $r_{t}^{*}$ the equilibrium rate of real interest, $(\pi _{t}-\pi _{t}^{*})$ deviation in inflation from the target rate, $(y_{t}-{\bar  y}_{t})$ difference in output from potential output.  Figure 1 shows the observed Federal Funds rate as compared to the interest rate prescribed by the above Taylor rule.  The similarity in the time series has been documented by macroeconomic literature in the past.\begin{equation}
i_{t}= \pi _{t} + r_{t}^{*} + a_{\pi }(\pi _{t} - \pi _{t}^{*})  + a_{y}(y_{t}-{\bar  y}_{t})  
\end{equation}
\begin{figure}[H]
\centering 
\caption{Nominal Interest Rate, Observed and Taylor Rule Prescribed}
\includegraphics[scale=0.75]{Taylor_i}
\end{figure}

\subsection{Econometric Specification}
\begin{equation}
i_{t}-\pi _{t}=r_{t}^{*}+a_{\pi }(\pi _{t}-\pi _{t}^{*})+a_{y}(y_{t}-{\bar  y}_{t})+\varepsilon_t 
\end{equation}
A simple Taylor Rule is specified in equation 2, where $i_{t}-\pi _{t}$ is the real interest rate, $r_{t}^{*}$ the equilibrium rate of real interest (here a constant term to be estimated), $(\pi _{t}-\pi _{t}^{*})$ deviation in inflation from the target rate,\footnote{The Federal Reserve has an official target rate of 2\% CPI, and we use this as our target.} $(y_{t}-{\bar  y}_{t})$ difference in output from potential output.  $a_{\pi }$, $a_{y}$ are parameters we estimate, and $\varepsilon_t$ is a regular error term distributed $N(0,\sigma^2)$.


The simple Taylor rule regression results are not reliable if any specific variables have order of integration not equal to zero -- we specifically look at the case that the real interest rate and output gap have order of integration of one.  It is not a common result finding the output gap has order of integration equal to one.\footnote{Although the interest rate, output gap, and inflation rate are highly persistent, we do not claim that they are non-stationary in general.  See \cite{rudebusch1993uncertain} for a discussion of this topic and \cite{osterholm2005taylor} for a full analysis of the Taylor Rule spurious regression problem - for which our methods are consistent with.}  Our sample sizes do not span all of the available data sample, instead are shorter subsamples corresponding to tenures of Federal Reserve Chairmen.  It follows that while stationarity may hold over the entire sample, it may not hold over our smaller subsamples, and we present results to justify these methods in Section 4.

When the above restrictions apply, the Granger Representation Theorem, as presented by \cite{engle1987co}, may be applied to our subsamples.  We allow for such interest rate smoothing by estimating the Taylor rule in the context of an error correction model.  This approach allows for the possibility that the funds rate adjusts gradually to achieve the rate recommended by a Taylor rule (Judd \& Rudebusch 1998\nocite{judd1998taylor}).
\begin{equation} i_t^* = \pi_t + r_t^* + \lambda_1 (\pi_t –- \pi^*) + \lambda_2(y_{t}-{\bar  y}_{t}) + \lambda_3(y_{t-1}-{\bar  y}_{t-1})+\varepsilon_t
\end{equation}
Equation 3 is similar to equation 2, but instead considers $i_t^*$, the  recommended nominal interest rate achieved by gradual adjustment (and not a direct Taylor rule), with a lagged output gap term.  This specification allows for multiple monetary policy targets, such as responding to only inflation, a nominal GDP growth target, inflation and output gap with varying or equal weights (as outlined by Judd \& Rudebusch 1998\nocite{judd1998taylor}).

Dynamics of adjustment of the actual level of the funds rate to $i_t^*$ are: 
\begin{equation}
\Delta i_t = \gamma(i_t^* –- i_{t-–1}) + \rho\Delta i_{t-1}
\end{equation}
In this Error-Correction Representation (equation 4) a change in interest rate depends on difference between recommended rate and previous period interest rate according to parameter $\gamma$, and momentum in interest rate change from previous time period according to parameter $\rho$. 
\begin{equation}
\Delta i_t = \gamma\alpha - \gamma i_{t-1} + \gamma(1+\lambda_1)(\pi_t-\pi^*) + \gamma\lambda_2(y_{t}-{\bar  y}_{t})+\gamma\lambda_3(y_{t-1}-{\bar  y}_{t-1})+ \rho\Delta i_{t-1}
\end{equation}
By substituting the previous equations, we obtain the equation
to be estimated in equation 5, where $\alpha= r^* - \lambda_1\pi^*$.  We cannot estimate both the equilibrium real interest rate and the inflation target simultaneously, and as such we use the official CPI target rate.  

% RESULTS
\section{Results}
\subsection{Chairman Greenspan, 1987.Q3-–2006.Q1}
\begin{table}[H]
\centering
\caption{Estimated Coefficients Under Assumption of Simple Taylor Rule}
\begin{tabular}{ccc|c}
$r_{t}^{*}$ & $a_{\pi }$ & $a_{y}$ & ${\bar R}^2$\\
\toprule
.34 & 0.38& 0.58***& 0.19\\
(0.36) & (0.23)& (0.17)  \\
\midrule
\bottomrule
\end{tabular}
\end{table}
Table 2 shows the results of OLS regression for an unadjusted Taylor Rule.  \cite{taylor1993discretion} famously suggested a rule of the form $a_{\pi }=a_{y}=\frac{1}{2}$, while this regression finds only the coefficient for an output gap significant, with a low explanatory value ($R^2$ of 0.17).  Augmented Dickey-Fuller test results on both output gap and real interest rate find presence of unit root, and so the variables are cointegrated.\footnote{ADF test statistics are -1.40 and -1.42 for real interest rate and output gap respectively over the restricted time period, indicating non-stationarity over this given time period and example.  It has already been mentioned that this result is not to be generalised to all output gap time series.} 

Table 3 presents the results of an adjusted Taylor rule for discretion over gradual change in interest rates and time series context.\footnote{Standard Errors are presented in brackets beneath estimated coefficients, and $^*$ $^*$ $^*$ designates significance at the 1\% level, $^*$$^*$ the 5\% level, and $^*$ the 10\% level.}  Regression A finds the lagged output gap as insignificant so that it is excluded in regression B.  Regression B explains 54 percent of the variation in interest rates over this time period (with an adjusted $R^2$ of 0.51), and is presented in full construction in equation 6.\footnote{It should be stressed that the coefficients represent the Federal Reserve's reaction to economic situations, and the coefficients in equation 6 are constructed from these reactions.}
\begin{table}[H]
\centering
\caption{Estimated Coefficients Under Adjusted Taylor Function, Equation B Omits Insignificant Output Gap Lag}
\begin{tabular}{ccccccc|ccc}
&$\alpha$ & $\gamma$ & $\lambda_1$ & $\lambda_2$& $\lambda_3$&$\rho $&${\bar R}^2$ \\
\toprule
A:&4.51***& 0.10*** & 0.35** & 1.80** &-0.51 & 0.45*** & 0.50 \\
&(0.13)&(0.028)&(0.061)&(0.090)&(0.090)&(0.098)&\\
\midrule
B:&4.60***&0.11***&0.28**&1.29***&&0.45***& 0.51 \\
& (0.12) &(0.027)&(0.060) &(0.038) &&(0.097) \\
\midrule
\bottomrule
\end{tabular}
\end{table}
\begin{equation}
\Delta i_t = 0.49^{***} +0.11^{***} i_{t-1} + 0.14^{**}(\pi_t-\pi^*) + 0.14^{***}(y_{t}-{\bar  y}_{t})+ 0.45^{***} \Delta i_{t-1} +\varepsilon_t 
\end{equation}
This specification shows for discretionary interest rate change according to gradual adjustment of the funds rate to a Taylor rule (as opposed to instantaneous change according to economic situations) with a highly significant value for momentum in interest rate change.  Each quarter, the interest rate adjusts enough to eliminate 11 percent of the difference between the lagged and optimal rate. Lastly, the estimated weight for output gap of 1.29 is much higher than the rate of 0.50 that Taylor originally prescribed.

\subsection{Chairman Bernanke 2006.Q2--2014.Q1}
\begin{table}[H]
\centering
\caption{Estimated Coefficients Under Assumption of Simple Taylor Rule}
\begin{tabular}{ccc|ccc}
$r_{t}^{*}$ & $a_{\pi }$ & $a_{y}$ & ${\bar R}^2$\\
\toprule
0.023&-2.46*** &1.00***& 0.87\\
(0.46)& (0.185) & (0.13) \\
\midrule
\bottomrule
\end{tabular}
\end{table}
Table 2 shows the results of OLS regression for an unadjusted Taylor Rule. This regression finds the coefficients for both inflation deviation and output gap significant, with an extremely high explanatory value ($R^2$ of 0.87).  Surprisingly the coefficient for reaction to inflation deviation is large (much greater than 1 in magnitude) and negative, which is a very counter-intuitive result in the context of a monetary policy reaction function.  Once again, Augmented Dickey-Fuller test results on both output gap and real interest rate find presence of unit root, indicating a spurious regression, and so the variables are cointegrated.\footnote{ADF test statistics are -2.38 and -1.51 for real interest rate and output gap respectively over the restricted time period indicating non-stationarity over this given time period and example.  Once again the previous comments against generalisation of this result hold.} 

Table 5 presents the results of an adjusted Taylor rule for discretion over gradual change in interest rates and time series context.  Regression A finds the vast majority of regressors to be insignificant, regression B considers the same model respecification as section 4.1, and regression C considers the model which drops all insignificant regressors. Regression C explains 47 percent of the variation in interest rates over this time period (with an adjusted $R^2$ of 0.43), and is presented in full construction in equation 7.
\begin{table}[H]
\caption{Estimated Coefficients Under Adjusted Taylor Function}
\centering
\begin{tabular}{ccccccc|ccc}
&$\alpha$ & $\gamma$ & $\lambda_1$ & $\lambda_2$& $\lambda_3$&$\rho $&${\bar R}^2$ \\
\toprule
A:&  1.85&0.063 &-1.59 & 2.46&-1.90 &0.40** & 0.34\\
& (0.31)&(0.064)&(0.056)&(0.10)&(0.11)&(0.17)&\\ \midrule
B:& 4.41 & 0.097*&-1.86 &1.14 & &0.46*** & 0.33\\
& (0.28)&(0.057)&(0.054)&(0.067)& &(0.16)&\\ \midrule
C:&&0.052**&&&&0.54***& 0.43 \\
&  &(0.023)&&&&(0.14)& \\
\midrule
\bottomrule
\end{tabular}
\end{table}
\begin{equation}
\Delta i_t = 0.52^{**} i_{t-1} + 0.54^{***} \Delta i_{t-1} +\varepsilon_t 
\end{equation}
The monetary policy reaction finds momentum in interest rate change highly significant.  Quaterly, the interest rate adjusts to eliminate five percent of the difference between the lagged and optimal rate -- an entire six percent lower than the change during Chairman Greenspan's tenure.  Most striking of the results of the reaction function are the insignificance of inflation deviation, output gap, and lagged output gap.  This result is very counterintuitive according previous literature for the Federal Reserve's reaction to current economic situations, not least the results presented by \cite{judd1998taylor} and the reaction function we estimated during the tenure of Chairman Greenspan.  This will be our main point of contention in the following discussion. 
% DISCUSSION
\section{Discussion}

The reaction functions estimated in sections 4.1 and 4.2 are noticeably different.  This difference mainly stems from the result that during Chairman Bernanke's tenure the reaction function finds the majority of regressors  with error correction/gradual adjustment in the Taylor Rule we present.  The result would imply that the Federal Reserve adjusted monetary policy according to neither the output gap nor inflation deviation from target, and supports out hypothesis that the Taylor rule paradigm is not the most effective way to consider monetary policy in the US from 2007 to 2014.

There are some key things to note about the time periods for the tenures of Chairmen Greenspan and Bernanke.  Firstly, Greenspan resided over the Federal Reserve when the concept of ``easy-money" in central banking was first coined.  It was the beginning of a systematic monetary policy switch from previously tight monetary policy, which combated primarily high inflation, to the Reserve being relied upon for economic stability, and so weighting output gap as relatively more important in monetary policy decisions.  This switch can be seen in the estimates for our reaction function for Chairman Greenspan's tenure compared to the same for Chairman Volcker tenure by \cite{judd1998taylor}, where a reaction function reveals a higher weighting by the Reserve to maintain low inflation for Volcker's tenure.

Chairman Greenspan's tenure is also documented as an era that monetary policy in the US was conducted very successfully: there were relatively few and short-lived recessions and there was longest recorded expansion in the history of the United States.  However, there are calls that the period of ``easy-money'' laid the seeds for the Great Recession in 2008.  During this time period there was a relatively low rate of variance in the output gap and inflation deviation (signaling a time period of economic stability), while no corresponding drop in variance of interest rate (signaling a higher level of discretion in monetary policy decisions).  It follows that a Taylor rule with error correction and discretion for gradual adjustment is a very effective way to consider monetary policy decision making at the Federal Reserve over the tenure of Chairman Greenspan.

On the other hand, Chairman Bernanke's tenure was remarkably different regarding both economic conditions and the nature of central banking itself.  For one, the Great Recession, which began a few quarters after Bernanke took office, led to historically high rates of unemployment and alarmingly low levels of inflation.  Figure 1 shows that for multiple quarters between 2008 and 2014, a simple Taylor rule would have prescribed a negative nominal interest rate.  The observed interest rate for this time was instead just above zero, and was kept at this rate from the end of 2008 to the middle of 2015.  These things would suggest that the Federal Reserve ran in to the zero lower bound in setting interest rates, and so could not change interest rates any further to reflect currently dire economic situations.  Our results support this hypothesis by showing a reaction function that considers neither of the historically most important economic indicators in monetary policy decisions.  

We can thus refute the standard Taylor rule and Taylor with discretion framework for effectively analysing monetary policy decisions at the Federal Reserve for Chairman Bernanke's tenure.  Since monetary policy decisions were constrained below by the zero lower bound, interest rate change could not have been decided by the standard Taylor rule approach of output gap and inflation deviation.  This time period may also be noted as one in which there were liquidity trap implications, so that changes in the nominal interest rate are expected to have lower than usual economic effects, as theoretically presented by \cite{krugman2000thinking}.  The Federal Reserve's actions in this time period reflected these considerations by embarking on central banking's most expansive policy advances in the form of quantitative easing and a sustained historically low interest rate.

It follows that the Federal Reserve's monetary policy decisions were less of the traditional Taylor rule variety during Chairman Bernanke's tenure and more of the unconventional interventionist approach that has characterised the new era of central banking, and that is still in full force in Europe and Japan.  The last item we would like to mention in our discussion of a Taylor rule and a Taylor rule with discretion for gradual change are Mr Bernanke's own interactions with the Taylor Rule after he left office.\footnote{When he was no longer federally constrained on expressing his views on monetary policy.}  \cite{bernanke_2016} presents a critique of the Taylor rule in a blog post for his new academic home, concluding that the Taylor rule's lack of complexity is its strongest weakness.  This in direct disagreement to Taylor's (1983) conclusion that simplicity and lack of discretion leads to greater credibility in monetary policy.  The debate stemmed from a discussion at an International Monetary Fund event, where Bernanke's opinions on monetary policy according to a Taylor rule became apparent, and his opposition to the idea that monetary policy should be decided directly according to economic situations without discretion\footnote{Discretion which we find includes being above and beyond that for error correction in gradual adjustment.} was made very clear.

% CONCLUSION
\section{Conclusion}
The estimates in this paper indicate that a Taylor rule reaction function is an effective way to analyse monetary policy during Chairman Greenspan's tenure as Federal Reserve Chairman, and captures some key elements in monetary policy decision making during this time.  Our regressions imply that changes in the interest rate are consistent with a monetary policy regime that aimed for stable inflation around the target and to minimise the output gap.  The Reserve's reaction differs from Taylor's (1993) original specification, specifically by reacting more strongly to the output gap and adjusting interest rates gradually rather than instantly to a Taylor rule specification.  Our results are consistent with results presented by \cite{judd1998taylor} before the end of Chairman Greenspan's tenure.

For Chairman Bernanke's tenure, we find that a Taylor rule reaction function is not an effective way to analyse monetary policy, and this reflects some key elements of the economic and monetary policy environment of that time period.  The reaction function did not find significant the key variables in a Taylor rule framework of deciding monetary policy.  We consider that the zero lower bound, Federal Reserve response by pursuing other areas of monetary policy, and Chairman Bernanke's tenure at the Reserve reflect this departure from precedence in monetary policy decision making.  These results break from the results of \cite{judd1998taylor} in regarding a Taylor rule for successive chairmen of the Federal Reserve.

Lastly, as of May 2017, The US seems to be on a path out of a period of central banking intervention and liquidity trap monetary policy.  However, the economic recovery has not been so strong around the world, especially in Europe where this new era of liquidity trap has yet to show signs of ebbing.  It remains to be seen whether the recovery in the US will continue so strongly in spite of international headwinds, and whether the era of monetary policy at the zero lower bound may be left behind in favour of the more conventional Taylor rule approach.  So that it remains to be seen whether the Taylor rule and Taylor rule with discretion for gradual adjustment will in the future ever again be an effective method to analyse monetary policy decision at the Federal Reserve.
\newpage
\bibliographystyle{agsm}
\bibliography{Bibliography}


\end{document}
